%%%%%%%%%%%%%%%%%%%%%%%%%%%%%%%%%%%%%%%%%
% University/School Laboratory Report
% LaTeX Template
% Version 3.1 (25/3/14)
%
% This template has been downloaded from:
% http://www.LaTeXTemplates.com
%
% Original author:
% Linux and Unix Users Group at Virginia Tech Wiki 
% (https://vtluug.org/wiki/Example_LaTeX_chem_lab_report)
%
% License:
% CC BY-NC-SA 3.0 (http://creativecommons.org/licenses/by-nc-sa/3.0/)
%
%%%%%%%%%%%%%%%%%%%%%%%%%%%%%%%%%%%%%%%%%

%----------------------------------------------------------------------------------------
%	PACKAGES AND DOCUMENT CONFIGURATIONS
%----------------------------------------------------------------------------------------

\documentclass{article}


\usepackage{amsmath} % Required for some math elements 
\usepackage{graphicx}
\usepackage{caption}

\setlength\parindent{0pt} % Removes all indentation from paragraphs

\renewcommand{\labelenumi}{\alph{enumi}.} % Make numbering in the enumerate environment by letter rather than number (e.g. section 6)

%\usepackage{times} % Uncomment to use the Times New Roman font

%----------------------------------------------------------------------------------------
%	DOCUMENT INFORMATION
%----------------------------------------------------------------------------------------

\space
\space
\title{Resultados Tarea 3 \\ Metodos Computacionales \\ 2017-19} % Title
\space
\space
\space
\author{Carol Vanessa \textsc{Barrera Lopez}} % Author name

\date{20 de Julio de 2017} % Date for the report

\begin{document}

\maketitle % Insert the title, author and date

\space
\space
\space
\space
\space
\space
% If you wish to include an abstract, uncomment the lines below
\begin{abstract}
En el presente PDF se presentan los resultados obtenidos luego de trabajar arduamente en la Tarea 3 de la materia Metodos Computacionales. 
\end{abstract}

%----------------------------------------------------------------------------------------
%	SECTION 1
%----------------------------------------------------------------------------------------

\section{Ecuaci\'on de onda en 2 dimensiones}
\subsection{Disperci\'on de onda en t=30}
\begin{centering}
\hspace{1cm}\includegraphics[width=8cm]{Onda30.pdf}
\hspace{20cm}\textbf{Fig 1.} Propagaci\'on de una onda en un contenedor de agua con una rendija (Pasados 30 segundos luego de la perturbaci\'on inicial).\centering
\end{centering}
\subsection{Disperci\'on de onda en t=60}
\begin{centering}
\hspace{1cm}\includegraphics[width=8cm]{Onda60.pdf}
\hspace{20cm}\textbf{Fig 2.} Propagaci\'on de una onda en un contenedor de agua con una rendija (Pasados 60 segundos luego de la perturbaci\'on inicial).\centering
\end{centering}



%----------------------------------------------------------------------------------------
%	SECTION 2
%----------------------------------------------------------------------------------------

\section{Sistema solar}

\hspace{1cm}\includegraphics[width = 10cm]{sistemaSolar.pdf}

\textbf{Fig 3.} Orbitas del sistema solar.\centering

\end{document}




